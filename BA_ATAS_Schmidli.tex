\documentclass[11pt,english,german]{report}

% Package import, Document Settings
\usepackage[a4paper,inner=3.5cm,outer=2.5cm]{geometry}
\usepackage[english,ngerman]{babel}
\usepackage[utf8]{inputenc}

% Packages
\usepackage{caption}
\usepackage{latexsym}
\usepackage[T1]{fontenc}
\usepackage{graphicx}
\usepackage{hyperref}
\usepackage{tabularx}
\usepackage{etoolbox}
\usepackage{fancyhdr}
\usepackage{amsthm}
\usepackage{mathtools}
\usepackage[xindy]{glossaries}
\usepackage{hyperref}
\usepackage{lastpage}

% clear default
\fancyhead{}
\fancyfoot{}

\pagestyle{fancy}
\fancyhf{}
\renewcommand{\headrulewidth}{0pt} % optional
\fancyfoot[L]{Kapitel: \nouppercase{\leftmark}}
\fancyfoot[R]{\thepage/\pageref{LastPage}}

% Redefine the plain page style, Chpater page
\fancypagestyle{plain}{%
  \fancyhf{}
  \renewcommand{\headrulewidth}{0pt} % optional
  \fancyfoot[R]{\thepage/\pageref{LastPage}}
}

\renewcommand{\chaptermark}[1]{\markboth{\MakeUppercase{#1}}{}}

% Glossar
\newglossaryentry{ATAS}
{
    name={ATAS},
    description={Aplinist Tracker \& Alerting System}
}
\newglossaryentry{entry}
{
	name={TTN},
	description={The Things Network, LoraWan Netzwerk}
}

\theoremstyle{definition}
\newtheorem{exmp}{Beispiel}[subsection]


\makeglossaries

\setcounter{secnumdepth}{2}
\setcounter{tocdepth}{1}

\begin{document}
\pagestyle{empty} %Keine Kopf-/Fusszeilen auf den ersten Seiten.
\begin{titlepage}
\begin{center}

% Oberer Teil der Titelseite:
\includegraphics[width=0.08\textwidth]{img/bfh_logo.png}\\[1cm]    
\textsc{\LARGE Bern University of Applied Sciences}\\[1.5cm]
\textsc{\Large Bachelor Thesis}\\[0.2cm]
\textsc{\Large Informatik}\\[0.5cm]

% Title
\newcommand{\HRule}{\rule{\linewidth}{0.3mm}}
\HRule \\[0.4cm]
{\huge Alpinist Tracker \& Alerting System}\\[0.3cm]
{\huge \bfseries  ATAS}
\HRule \\[2cm]

\includegraphics[width=0.2\textwidth]{img/atas_logo.png}\\[3cm]    

% Author und Lehrer
\begin{minipage}{0.32\textwidth}
\begin{flushleft} \large
\emph{Autor:}\\
Martin \textsc{Schmidli}\\
\end{flushleft}
\end{minipage}
\hfill
\begin{minipage}{0.32\textwidth}
\begin{flushleft} \large
\emph{Dozent:} \\
Mohamed \textsc{Mokdad}
\end{flushleft}
\end{minipage}
\hfill
\begin{minipage}{0.32\textwidth}
\begin{flushleft} \large
\emph{Experte:}\\
Daniel \textsc{Voisard}\\
\end{flushleft}
\end{minipage}

\vspace{20mm}

% Unterer Teil der Seite
Bern, {\large \today}
\end{center}
\end{titlepage}
\pagestyle{fancy}


\tableofcontents


\chapter{Einleitung}
Stellen Sie sich ein kleines mobiles Gerät vor, nachfolgend Tracker genannt, welches Skifahrern, Wanderer usw. abgegeben werden kann. Das Gerät sendet die Position der Person an einen Empfänger, nachfolgend Gateway genannt. Der Gateway wird bei der Talstation oder im nächsten Bergdorf montiert. Der Gateway sendet die empfangenen Daten der Tracker an eine zentrale Stelle irgendwo im Internet. Die Administratoren des Systems, beispielsweise die Rega, können schlussendlich über eine Webseite die aktuelle Position der Personen in den Bergen mitverfolgen und überwachen.

\section{Aufgabenstellung}
Ziel der Bachelorarbeit soll es sein, dass während dem Projekt2 aufgebaute System intensiv zu testen und zu verbessern. Die Arbeit umfasst zwei Hauptaufgaben. Zum einen Teil soll der Tracker in realer Umgebung getestet werden bspw. während einer Wanderung. Die Messdaten werden erfasst und analyisert. Zum anderen soll der Tracker Prototyp verbessert werden; Reichweite, Aufbau, Verlässlichkeit usw. Die dazu erforderlichen Massnahmen werden während der Bachelorarbeit spezifiziert und umgesetzt. Am Ende der Arbeit soll eine klare Aussage gemacht werden können ob das erdachte System praxistauglich ist.


\section{Rahmenbedingungen}
Die Rahmenbedingungen dieses Projektes wurden gemeinsam mit dem Betreuer definiert
\begin{itemize}
\item Die Bachelorarbeit baut auf der Arbeits des Projekt 2 auf. Die erstellete Hard und Software wird weiterverwendet.
\item The Things Network dient als Plattform für die Kommunikation unter den Komponenten.
\item Als Programmiersprache soll C/C++ verwendet werden
\item Auf den Einsatz eines Betriebsystememes auf dem Tracker Node soll verzichtet werden
\end{itemize}

\section{Abgrenzungen}
Sie als Leser denken nun vielleicht: "Das ist überflüssig. Ich habe doch mein Handy für so etwas? Schreib doch eine App!". Diese Aussage mag auf die touristischen Wander- und Skigebiete zutreffen. Der Empfang ist meistens wunderbar. Verlasssen wir diese "sicheren" Orte und bewegen uns aber in höheren Lagen, wird der Empfang mit dem Handy immer schlechter oder existiert gar nicht. \bigskip \\
Aus diesem Grund mussten für dieses Projekt andere Technologien gefunden werden. In diesem Projekt werden folgende Technologien eingesetzt:
MQTT, LoRa, LoRaWAN\bigskip \\
Die verwendetedn Technologien werden in einem späteren Abschnitt genauer erklärt.

\chapter{Systembeschreibung}
Um die nachfolgenden Kapitel zu verstehen, ist es sehr wichtig, sich mit der im Projekt 2 erstellten Systemarchitektur und die Terminologie vertraut zu machen. Die Architektur wurde im vergleich zum Projekt 2 um einige Komponenten ergänzt.

\section{Grobe Systemarchitektur}
Dieser Abschnitt bietet Ihnen einen groben Überblick über die Benutzer, die Komponeneten und deren Beziehung innerhalb des ATAS Systems. Alle Einheiten werden auf den nachfolgenden Seite detailiert beschrieben.

\newpage
\subsection{Aufbau}
\subsection{Diagram}

\subsection{Benutzer}
Die nachfolgenden Gruppen wurden als Benutzer des Systems identifiziert.
\subsubsection{Aplinist}
Alpinist wird als Generalisierung für Personen, welche sich in den Bergen aufhalten verwendet. Dazu gehören bspw. Wanderer, Skifahrer, Bergbauer usw.
\subsubsection{Überwacher}
Rettungsdienste wie bspw. die Rega oder die AirGlacier. Spitäler oder die lokalen Tourismusbehörden.

\subsection{Komponenten}
Das ATAS System besteht aus den nachfolgenden Komponenten. Komponenten sind als Hardware und Software zu verstehen.
\subsubsection{Tracker}
Stellen Sie sich ein kleines mobiles Gerät vor, nachfolgend genannt Tracker. Der Alpinist trägt den Tracker bei sich bspw. in einem Rucksack.\bigskip \\ 
Der Tracker verfügt über ein Display, Taste, GPS Modul, Kommunikationsmodul (Lora) und ein Lautsprecher.
\begin{itemize}
\item
Taste: Mit dem Druck auf den Knopf können die Überwacher über eine Notsituation aufmerksam gemacht werden. Bspw. Wenn der Alpinist einen Unfall hatte und nun bewegungsunfähig ist.
\item
Lautsprecher: Über den Lautsprecher kann der Alpinist über eine Gefahrenquelle mit einem akustischen Signal aufmerksam gemacht werden. Bspw. Wenn sich der Alpinist in einem Bereich am Berg mit erhöhter Gefahr für Lawinen aufhält. Der Lautsprecher dient zur Information, \textbf{dass ein Problem besteht}.
\item
Display: Über ein Display können mehr Informationen zum Tracker und den Gefahrenzonen angezeigt werden. Das Display dient zur Information, \textbf{was für ein Problem besteht}.
\item
GPS Modul: Der Tracker verfügt über ein GPS Modul. Mit dem GPS Modul kann die Position des Tracker auf der Erde ermittelt werden.
\item 
Lora Modul: Mittels Lora Modul können Daten an einen Empfänger, nachfolgend \textbf{Atas-Gateway} genannt, gesendet werden.
\end{itemize}

\subsubsection{Gateway}
Ein Gerät welches mit den Trackern bidirektional kommuniziert. Der Gateway wird an einem Ort nahe den Bergen installiert. Als mögliche Orte kommt ein Dorf im Tal oder eine Talstation in Frage.\bigskip \\
Der Gateway senden die empfangenen Daten an eine zentrale Datenmanager, nachfolgend \textbf{Broker} genannt.

\subsubsection{Broker}
Sammelt und speichert Daten. Der Broker kommuniziert mit den Gateways. Zusätzlich bietet der Broker ein Interface zum Abfragen und senden von Daten.

\subsubsection{Webapplikation}
Die Notfalldienste resp. Überwacher nutzen eine Webapp zum interagieren mit dem ATAS System. Die Webseite bietet folgende Funktionalitäten
\begin{itemize}
	\item
		Zeigt die Position der Tracker auf einer Karte an 
	\item
		Mittels der Webapp können Gefahrenzonen hinterlegt werden.
\end{itemize}

\subsubsection{Business Logik}
Software welche berechnet, ob sich ein Tracker in einer Gefahrenzone aufhält. Wenn sich der Tracker in einer Gefahrenzone aufhält, senden das System eine Alarm an den Tracker. Das Signal gelangt via Broker und dem Gateway zum Tracker.

\section{Anwendungsfälle}
Dieses Abschnitt soll Ihnen erläutern, wie das ATAS System genutzt werden kann.
\begin{itemize}
	\item 
	Wenn in den Bergen eine Lawine ausgelöst wird, kann deren Position mit den Positionen der Tracker verglichen werden. Befindet sich ein Tracker in der Gefahrenzone, kann die Rettungsmannschaft sofort reagieren und ausrücken. Dieser Prozess läuft sofort ab und kann dabei die Überlebenschance der Opfer erhöhen. 
	\item 
	Wurden Personen von der Lawine begraben und der Tracker hat überlebt, könnte das Gerät weiterhin die Position senden. Kombiniert mit modernen Lawinensuchgeräten können die Einsatzkräfte gezielter nach Überlebenden suchen. Wenn keine Übertragung mehr möglich ist, wissen die Überwacher zumindest den letzten Aufenthaltsort. Das ATAS System hat nicht das Ziel Lawinensuchgeräte zu ersetzen, es ist eher als Ergänzung zu verstehen. 
	\item 
	Bewegt sich Tracker auf eine Gefahrenzone zu, beispielsweise ein Gebiet mit erhöhter Steinschlaggefahr, könnte die Person frühzeitig davor gewarnt werden. 
	\item 
	Ist es zu einem Unfall gekommen, kann der Alpinist mittels Tracker ein Notsignal absetzen. Dazu muss der Benutzer nur auf den Notfallknopf drücken.
\end{itemize}

\chapter{Vorgehen}
Das Bachelorarbeit wird in 2 Hauptaufgaben aufgeteilt.\textbf{ Testing} und das erstellen eines zweiten Prototyps nachfolgend genannt \textbf{Prototyp 2}.
\section{Testing}
Ein Teil des Systemaufbaus aus dem vorgängigen Projekt soll getestet werden. In diesem Kapitel wird genau beschrieben, wie beim Testing des Systems vorgegangen wird.\\[0.3cm]
In der nächsten Abbildung sind die geplanten Schritte aufgeführt.
\begin{figure}[h]
	\centering
	\includegraphics[width=\textwidth]{img/projectFlow_testing.jpg}
	\caption[Flowchart Testing]
	{Ablauf Testing}
\end{figure}
\\ 
Die einzelnen Schritte werden auf den kommenden Seiten detailliert erklärt.
\newpage
\subsection{Definition}
Es muss definiert werden
\begin{itemize}
	\item welche Komponenten geprüft werden sollen\\[0.3cm]
	Während der Arbeit soll die Kommunikation zwischen Atas-Node und Atas-Gateway getestet werden.
	\item welche Aspekte der Komponente sollen betrachtet werden\\[0.3cm]
	Der Fokus liegt auf der \textbf{Zuverlässigkeit} der Übertragung. Mich interessiert, wie sicher es ist, dass eine Meldung ihr Ziel erreicht. 
	\item mit welchen Mitteln resp. Messinstrumenten gemessen wird\\[0.3cm]
	Das TTN Netzwerk bietet uns die Möglichkeit mittels Dashboard einzusehen, wann welche Pakete auf dem Gateway eintreffen. So können wir prüfen ob die Verbindung zwischen Gateway und Tracker funktioniert.\bigskip
	
	\begin{minipage}{\linewidth}
		\centering
		\includegraphics[width=\linewidth]{img/ttn_messages}
		\captionof{figure}{TTN Messages}
	\end{minipage}
	
	\item welche Daten sollen untersucht werden\\[0.3cm]
	Für die Messungen sind mehre Daten von Interesse. Dazu gehören 
	\begin{itemize}
		\item Signalstärke
		\item Wie viele Pakete gehen verloren
		\item Distanz der Übertragung
		\item Sichtverhältnisse zwischen Sender und Empfänger
	\end{itemize}
\end{itemize}

\subsubsection{Schlussfolgerung}
Abgeleitet aus der vorhergehenden Definitionen ergibt sich die folgende Aufgabenstellung im Bereich Testing. Es muss eine Testumgebung aufgebaut und die Verbindung getestet werden. Um von den bestehenden TTN Gateways unabhängig zu sein, wird ein \textbf{eigener Gateway} aufgebaut. Anschließend werden mehrere Wanderungen auf verschiedenen Routen durchgeführt. Die Verbindung wird aufgebaut und mittels TTN Dashboard geprüft.

\newpage
\subsection{Standort}
Ausgehend von den vorangegangen Definitionen muss folgendes definiert werden.
\begin{itemize}
	\item Es muss ein Ort gefunden werden, wo der Gateway platziert werden kann.\\[0.3cm]
	Der Standort kann frei gewählt werden. Einzige Voraussetzungen ist ein Zugang zum Internet. Der Gateway empfängt die Daten der Endgeräte und muss diese dem TTN Netzwerk über das Internet weiterleiten können. Eine erhöhte Position ist von Vorteil.
	\item Für die Tests sollten mind. 3 Routen/Wanderungen definiert werden. Dies garantiert eine vernünftige Anzahl an Messwerten.
	\item Auf dem Pfad selbst werden mind. 5 Orte definiert wo die Verbindungstests durchgeführt werden.
\end{itemize}
	
\subsection{Messungen}
Die vorgängig definierten Routen werden abgelaufen und die Messungen werden an den definierten Standorten durchgeführt.

\subsection{Auswertung}
Die erhobenen Daten werden analysiert und kategorisiert. Messungen sollen in 3 Kategorien eingeteilt werden.
\begin{enumerate}
	\item Kein Sichtkontakt zwischen Empfänger und Sender.
	\item Wenig Sichtkontakt resp. teilweise blockiert. Bspw. durch Bewaldung oder Steinformationen
	\item Direkter Sichtkontakt
\end{enumerate}

\newpage
\section{Prototyp 2}
Der im Projekt 2 erstellte Prototyp wurde mit sehr simplen vorgefertigten Elektronikkomponenten umgesetzt. Ziel dieser Arbeit war es in möglichst kurzer Zeit einen Prototypen aufzubauen. Während dieser Arbeit sollen nun neue Komponenten evaluiert und ein zweiter Prototyp erstellt werden. Der Ablauf dieser Phasen sieht wie folgt aus:\\[0.3cm]

\begin{figure}[h]
	\centering
	\includegraphics[width=\textwidth]{img/projectFlow_hardware.jpg}
	\caption[Flowchart Prototyp 2]
	{Ablauf für den Aufbau des zweiten Prototyps}
\end{figure}

Die einzelnen Schritte werden auf den kommenden Seiten detailliert erklärt.

\subsection{Evaluation Hardware}
Auf Grundlage der bestehenden Prototyps soll neue Hardware evaluiert werden. Der Funktionsumfang des Prototyps soll gleichwertig bleiben. Sind die Komponenten erst definiert, wird das Material bestellt.

\subsection{Zusammenbau Hardware}
Die Komponenten werden zusammengebaut und untereinander verbunden. Zur besseren Übersicht wird ein Schema  der Elektronik erstellt.

\subsection{Software Module}
Der bestehende Sourcecode des Prototyps 1 kann leider nicht eins zu eins übernommen werden. Die Software ist sehr stark für die Raspberry Pi Umgebung angepasst und ist daher nicht mit dem ESP32 Umfeld kompatibel. Die Software muss portiert werden. Die Logik der Software kann übernommen werden.\\[0.3cm]
Pro Hardware Modul wird ein separates Software Modul erstellt.

\subsection{Finalisierung}
Die Erstellten Softwaremodule werden miteinander verknüpft und die Software fertiggestellt. Abschließend werden einige Testszenarien durchgeführt. 

\subsection{Begründung}
Es gibt diverse Gründe für diesen Schritt
\begin{enumerate}
	\item Auf dem Prototyp 1 wird ein vollwertiges Betriebssystem (Raspian, Linux Derivat) eingesetzt. Einige Funktionen bspw. das Lesen der GPS Daten werden vom System sehr vereinfacht. Dies ermöglichte mir einen raschen und unkomplizierten Aufbau der Software. Obschon ein solches 'High Level' OS eine grosse Hilfe in der Entwicklung darstellt, ist es in der Industrialisierung der Lösung eher hinderlich. Der Einsatz eines OS hat diverse Nachteile
	\begin{itemize}
		\item Vom OS selbst benötigen wir für den Betrieb der Atas-Node Software nur einen Bruchteil der Funktionalität. Speicher, Memory und Systemperformance wird für Systemprozesse verschwendet. 
		\item Durch die hohe Auslastung des Systems werden stärkere Prozessoren benötigt. Der Energieverbrauch der Lösung steigt.
		\item Mehr Software bedeutet Zwangsläufig mehr Fehlerquellen. Obschon das eingesetzte System (Raspian) als Stabil gilt, kann es aufgrund der Softwarekomplexität zu Fehlern kommen.
		\item Durch den Einsatz eines OS reduzieren wir die Sicherheit des ganzen Systems. Je mehr Interfaces ein System anbietet, desto anfälliger wird es für den unerlaubten Zugriff.
	\end{itemize}
	\item Der Prototyp 1 bietet und viele Anschlussmöglichkeiten. Bspw. für ein Display, USB, oder eine RJ45 Buchse. Funktionen die wiederum welche Entwicklung stark vereinfachen, aber den Energieverbrauch erhöhen und die Grösse des Systems unnötig vergrößert. 
	\item Die Verbauten Komponenten sind nicht für eine Industrialisierte Lösung geeignet. Man könnte sogar von einem 'Bastelprojekt' sprechen. Die Komponenten hindern das Projekt daran eine professionelle Form anzunehmen.
\end{enumerate}
\newpage
Aus meiner Sicht bildet während der Arbeit zu konstruierende Prototyp 2 nur einen Zwischenschritt bis zu Finalen Version. Eine Visualisierung meiner Vorstellung:\\
% TODO BILD
Mein Ziel ist es mit dem Umstieg auf anderen Komponenten, den Grundstein für den Prototypen 3 zu legen. Der Prototyp 3 ist nicht Bestandteil dieser Arbeit.

%- Portieren der Software auf ES32
%- TTN als Platform
% SPI Schnittstelle analyse, was schickt die Lorwan library
%- Bitrate + laufzeit Analyse
%- Minimierung 

\chapter{LoraWan}
Um das nachfolgende Testing durchzuführen, musste ich mir mehr Wissen zum Thema LoraWan aneignen. Das im Projekt 2 aufgebaute System ist lauffähig, verwendet aber die Standardeinstellungen welche ich einfach übernommen habe. Das Wissen, ob und wie man das System optimieren könnte ist nicht vorhanden.
\section{Begriffserklärung}
Zunächst ist es nötig einige Begriffe aus dem LoraWan Umfeld zu verstehen\\[0.5cm]
\begin{tabularx}{\textwidth}{ l|X }
	\textbf{Begriff} & \textbf{Erklärung} \\ \hline
	Endgerät & LoraWan Endgerät. Atas-Node ist ein LoraWan Endgerät.\\ \hline
	AppEUI& Eindeutige ID einer Applikation. 64Bit lang. Wird genutzt um ein Endgerät einer Applikation zuzuweisen.\\ \hline
	DevEUI& Eindeutige ID eines Endgerätes. \\ \hline
	DevAddr & Logische Geräte Adresse, analog einer IP Adresse in einem IP Netzwerk.\\ \hline
	NetSKey & Wird für die Verschlüsselung der Kommunikation zwischen Endgerät und Netzwerkanbieter bspw. TTN verwendet. Mittels Schlüssel kann die Datenintegrität sichergestellt werden d.H. Manipulierte Nachrichten werden erkannt.\\ \hline
	AppSKey & Mittels AppSKey werden die Nutzdaten separat nochmals verschlüsselt. Nur der Nutzer des Netzwerks kann die Daten lesen das Transportnetzwerk bspw. TTN kann diese nicht einsehen.\cite{ttnsecurity}\\ \hline
	Downlink & Kommunikation vom Gateway zu Endgerät\\ \hline
	Uplink & Kommunikation vom Endgerät zum Gateway
\end{tabularx}

\newpage
\section{Security}
\subsection{Engerät Verbindungsmethoden}
OTAA (Over-The-Air-Activation) und ABP (Activation By Personalization) sind Verbindungsmethoden für LoraWan Endgeräte in einem LoraWan Netzwerk.\cite{jaguar}.\\[0.3cm]
Die beiden Methoden bieten verschiedene Vor und Nachteile.

\subsubsection{ABP}
ABP ist das simplere aber \textbf{weniger sichere} Verfahren. Die Adresse DevAdr sowie die Schlüssel NetSKey und des AppSKey werden für jedes Endgerät \textbf{einmalig} generiert und fix auf dem Endgerät hinterlegt. Gelingt es einem Angreifer sich Zugang zum Endgerät zu verschaffen, kann er die Schlüssel stellen, ein zweites Gerät am Netzwerk registrieren, die Identität des Originals annehmen und damit die Daten verfälschen.

\subsubsection{Kommunikation}
Das Gerät muss sich nicht am Netzwerk anmelden. Das Endgerät kann direkt Daten senden. Empfängt ein Gateway die Daten, werden die Keys geprüft und die Kommunikation entsprechend an oder abgelehnt.

\subsubsection{OTAA}
Das Gerät muss sich am Netzwerk anmelden. Dieser Vorgang wird auch Join Prozedur genannt. Die DevAddr sowie die Keys NetSKey und AppSKey werden bei jeder Aktivierung des Gerätes \textbf{neu generiert} und an das Endgerät übertragen. Dieses Verfahren ist sicherer. Da es sich hier um eine bidirektionale Kommunikation handelt, müssen die Komponenten (Gateway \& Endgerät) Downlinks unterstützen.\\[0.3cm]
Das Installieren (Deployment) von Endgeräten wird vereinfacht. Das generieren vom AppSKey und NetSKey pro Endgerät entfällt. 

\newpage
\subsubsection{Kommunikation}
Die nachfolgende Grafik zeigt auf, wie der Kommunikationsaufbau via OTAA abläuft \cite{jaguar}.
\begin{figure}[h]
	\centering
	\includegraphics[width=0.9\textwidth]{img/otaa_schema.png}
	\caption[OTAA Ablauf]
	{OTAA Ablauf}
\end{figure}

\begin{enumerate}
	\item Gerät sendet einen Join Request.
	\item Die Gateways empfangen die Anfrage
	\item Das LoRaWan Netzwerk  bspw. TTN prüft die Angaben
	\item Session Keys werden generiert
	\item Keys werden an das Endgerät gesendet und für die zukünftige Kommunikation verwendet
\end{enumerate}

\newpage
\subsection{Schlussfolgerung}
Obschon die Sicherheit während diesem Projekt nicht im Vordergrund steht, ist es sicher zukunftsorientierter direkt OTAA einzusetzen. Systeme welche zum Schutz von Personen eingesetzt werden, müssen so sicher wie nur möglich konzeptioniert sein.\\[0.3cm]
Sowohl Endgerät als auch Gateway unterstützen Downlinks, damit gibt es keine technische Hürde der den Einsatz von OTAA verhindern würde.

\section{Frame Counters}
Ein Endgerät besitzt zwei Zähler (FCntUp, FCntDown). Die Zähler werden bei einer Downlinkmessage (FCntDown) resp. Uplinkmessage (FCntUp) erhöht (+1).\\[0.3cm]
Wird eine Reply Attacke durchgeführt d.H. ein Paket nochmals gesendet, wird dieses Paket vom System verworfen. Dies geschieht deshalb, weil das System bereits eine Nachricht mit dem gleichen Framecounter erhalten hat.\\[0.3cm]
Diese zusätzliche Information wird mir beim Testen der Übertragung überaus nützlich sein. Anhand der Nummerierung, kann sehr schnell erkannt werden, ob ein Paket nicht sauber empfangen werden konnte. Bspw. Erhalten wir die Pakete 4,5,7 $\rightarrow$ Paket 6 wurde nicht korrekt übertragen.

\section{Adaptive Data Rate (ADR)}
Ist Adaptive Data Rate aktiv (ADR), entscheidet das LoraWan Endgerät selbständig ob und wie die Kommunikation optimiert werden soll. Gemäss TTN \cite{ADRTTN} sollte ADR nur für statische Nodes oder für mobile Nodes zur Erkennung von Stops (keine Bewegung) verwendet werden. Gemäss TTN ist ADR also nicht unbedingt für die Atas-Tracker geeignet. Die Tracker sind ja meistens in Bewegung bspw. durch Laufen, Ski fahren usw. Dennoch wurde mein Interesse geweckt und ich wollte genauer wissen warum dieses Feature nicht empfohlen wird.\\[0.3cm]
Sobald ein LoraWan Endgerät dem Netzwerk mitteilt, dass es ADR verwenden möchte, beginnt das TTN Daten aufzuzeichnen. Die letzten 20 Messdaten die der Node gesendet hat werden analysiert. Es folgt ein Beispiel von der TTN Webseite:



\chapter{Testing}
\section{Aufbau Gateway}
\subsection{Installation Software}
Die Installation der Gateway Software wird von TTN bereitgestellt und von der Community aktiv verwaltet und verbessert. Die TTN Community Gruppe aus Zürich hat ein sehr simples Installationsskript zusammengestellt um die Gateway Software automatisiert zu installieren.

Vorgehen auf dem Gateway
\begin{enumerate}
	\item Paketdefinitionen geupdated
	
\end{enumerate}
Code:
apt-get update
apt-get upgrade

%sudo dpkg-reconfigure locales // de_CH
sudo dpkg-reconfigure tzdata // Europe/Zurich

sudo raspi-config // Enable SPI

git clone https://github.com/ttn-zh/ic880a-gateway.git ~/ic880a-gateway
cd ~/ic880a-gateway

change EUI Source to Wireless
%GATEWAY_EUI_NIC="wlan0"

sudo ./install.sh spi


\chapter{Prototyp}
Während  im Projekt 2 aufgebaute Prototyp 

- prot schnell gebaut, bestehende koponenten, lego
- Gründe: Energie, Komponenten volle kontrolle, 

\section{Entwicklungsumgebung}
Duch den Umstieg vom Rapberry Pi auf den Mikrocontroller ESP32 hat sich im Bereich der Entwicklungsumgebung einiges verändert. Die ESP32 Plattform unterstützt, Stand heute (16.11.2017), zwei Entwicklungsumgebungen. Das wäre einerseits ESP-IDF (IoT Development Framework) sowie Arduino\cite{espidfarduino}. Die beiden Umgebungen sind miteinander kombinierbar. So kann in der IDF Umfeld die Arduino Unterstützung als Komponente importiert werden. Bestehende für Arduino konzipierte Bibliotheken können dadurch ebenfalls verwendet werden. Ich habe mich in Absprache meines Betreuers für die ESP-IDF Umgebung entschieden und mein System entsprechend vorbereitet. Die Installation der Entwicklungsumgebung ist nicht schwer, bedingt aber einige manuelle Schritte. Die Umgebung besteht aus einer Reihe von Tools (Toolchain) zum Kompilieren, Fehlersuche und dem Flashen des Mikrocontrollers.\\[0.3cm]
Ich verzichte an dieser Stelle auf eine detaillierte Auflistung aller Installationsschritte. Die Installation wurde gemäss der Anleitung auf der Webseite der Firma Espressif durchgeführt \cite{espidfinstallation}.\\[0.3cm]
Als Hostsystem wurde ein macOS System verwendet. Damit das System den ESP korrekt erkennen konnte, musste ein zusätzlicher Treiber installiert werden \cite{espidfdriver}. 
\newpage

\section{Hardware}
Für den Aufbau des zweiten Prototypen wurde die nachfolgende Hardware verwendet. 

\begin{table}[htbp]
    \centering
	\begin{tabularx}{\textwidth}{lll}
		Name & Gerätetyp & Bild \\ \hline
		Espressfif ESP32 & Entwicklungbaord & to be created\\ \hline
	\end{tabularx}
\end{table}

\newpage
\section{Schema}
Die Kommunikation unter den Komponenten wurde im Laufe der Projektarbeit  für mich immer unübersichtlicher. Um mir die Arbeit zu erleichtern, habe ich ein Schema %TODO welches Schema) 
erstellt. Vor der Arbeit hatte ich noch nie ein Schema gezeichnet und musste mich zuerst einige Stunden in die Materie einarbeiten. Auf anraten einiger Kollegen, mit Erfahrung im Elektronik Umfeld, sowie der frei verfügbaren Lernressourcen und Vorlagen, fiel meine Wahl auf das Programm \textbf{Eagle} der Firma Autodesk \cite{autodesk}. Dank der guten Dokumentation und den Lernvideos auf diversen Plattformen bspw. Youtube kam ich sehr schnell vorwärts. \\[0.3cm] 
Ich begann damit Vorlagen der Elektronischen Bausteine/Chips finden. Die Vorlagen zu Lora, GPS, ESP, Schalter und Piezo Lautsprecher konnte ich teilweise vom Hersteller oder von der Community herunterladen und in Eagle importieren. Für das Display musste ich ein bestehende Vorlage eines Displays anpassen. Ich habe dabei die Größe und die Pinbelegung verändert, damit das Element so ähnlich wie nur möglich wie das verwendete Display (Waveshare eInk 200x200) aussieht. \\[0.3cm]
Das Schema ist auf der nachfolgenden Seite abgebildet.
\newpage

\begin{figure}[h]
	\centering
	\includegraphics[width=\textwidth]{img/prototyp_schema.png}
	\caption[Prototyp2 Schema]
	{Prototyp Schema}
\end{figure}

\newpage
\subsection{Mikrocontroller - Pinbelegung}
Die nachfolgende Pinbelegung wurde verwendet um den ESP32 mit den übrigen Komponenten zu verbinden. Zu meiner Überraschung sind die Pins nicht fix durch den ESP32 festgelegt. Die Pins können meist für jegliche Funktion bspw. PWM, SPI verwendet werden. Durch diese Freiheit wird die Verkabelung massiv vereinfacht und die Anzahl der möglichen Fehlerquellen wird reduziert.\\[0.5cm]
%Todo tabular header
\begin{tabularx}{\textwidth}{ l|l|X }
		MCU Pin & Modul & Funktion\\ \hline
		IO02 & Lora & SPI ChipSelect (CS)\\\hline
		IO04 & Lora + Display& SPI Busy\\ \hline
		IO05 & Displayl& SPI ChipSelect(CS)\\\hline
		IO15 & Switch & -\\ \hline
		IO16 & Lora + Display & SPI Reset\\\hline
		IO17 & Display&  DataCommand (DC)\\ \hline
		IO18 & Lora + Display & SPI Clock (CLK)\\ \hline
		IO19 & Lora + Display & Master In Slave Out (MISO) \\\hline
		IO21 & Piezo Speaker & - \\ \hline
		IO23 & Lora + Display & SPI Master Out Slave In (MOSI) \\\hline
		IO26 & GPS & UART RX\\ \hline
		IO27 & GPS & UART TX\\ \hline
		IO32 & Lora & DIO0\\ \hline
		IO33 & Lora & DIO1\\ \hline
\end{tabularx}

\chapter*{Selbständigkeitserklärung}
\label{chap:selbstaendigkeitserklaerung}

\vspace*{10mm} 

Ich bestätige, dass ich die vorliegende Arbeit selbstständig und ohne Benutzung anderer als der im Literaturverzeichnis angegebenen Quellen und Hilfsmittel angefertigt habe. Sämtliche Textstellen, die nicht von mir stammen, sind als Zitate gekennzeichnet und mit dem genauen Hinweis auf ihre Herkunft versehen. 

\vspace{15mm}

\begin{tabbing}
xxxxxxxxxxxxxxxxxxxxxxxxx\=xxxxxxxxxxxxxxxxxxxxxxxxxxxxxx\=xxxxxxxxxxxxxxxxxxxxxxxxxxxxxx\kill
Ort, Datum:\> Bern, 15.01.2018 \\ \\
Namen Vornamen:\> Martin Schmidli  \\ \\ \\ \\ 
Unterschriften:\> ...................................... \\
\end{tabbing}

\chapter*{Anhang}
\section{Software}
Es wird kein Code direkt an dieses Dokument angehängt. Zur Verwaltung des Sourcecodes wurde die Plattform Github verwendet. Jegliche Commits vor dem 16.09.2017 gehören zur Projekt 2 Arbeit. Commits nach diesem Datum wurden im Zuge der Bachelor Thesis erstellt. 
Folgend Sie den Links um den Sourcecode einzusehen.
\subsection{Atas-Webapp}
\url{https://github.com/schmm2/atas-webapp}
\subsection{Atas-Service}
\url{https://github.com/schmm2/atas-service}
\subsection{Atas-Node}
Software welche auf dem 1. Prototypen eingesetzt wird\\
\url{https://github.com/schmm2/atas-node}
\subsection{Atas-Node2}
Software welche auf dem 2. Prototypen eingesetzt wird\\
\url{https://github.com/ATAS-Group/atas-node2}



\printglossaries

\listoffigures

\begin{thebibliography}{1}
	\bibitem{ttnsecuirty} \url{https://www.thethingsnetwork.org/wiki/LoRaWAN/Security} 27.07.2017
	\bibitem{jaguar} \url{https://www.jaguar-network.com/en/news/lorawan-in-a-nutshell-2-internet-of-things-iot}, 27.07.2017
	\bibitem{ADRTTN} \url{https://www.thethingsnetwork.org/wiki/LoRaWAN/ADR} 27.07.2017
	\bibitem{autodesk} \url{https://www.autodesk.com/products/eagle} 27.07.2017
	\bibitem{espidfarduino} \url{http://iot-bits.com/documentation/esp32-tutorial-and-example-programs/} 20.12.2017
	\bibitem{espidfinstallation} \url{https://dl.espressif.com/doc/esp-idf/latest/get-started/index.html} 27.12.2017
	\bibitem{espidfdriver} \url{https://www.silabs.com/products/development-tools/software/usb-to-uart-bridge-vcp-drivers} 27.12.2017
\end{thebibliography}

\end{document}